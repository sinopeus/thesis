%************************************************
\chapter{Preface}
\label{chap:preface}
\mtcaddchapter
%************************************************

The papyri are an invaluable resource for documenting the history and evolution
of the Greek language. The recently published \emph{The Language of the Papyri}
[\cite{lpapyri}] has placed the spotlight firmly on the potential of this field
while at the same time pointing out the regrettable lack of recent scholarly
work available in it.

Though there is now a near-comprehensive collection of texts available that is
well-formatted and can easily be converted into an adequate corpus for
linguistic research, comparatively few scholars are interested in exploiting
the available resources; and none, to my knowledge, have attempted to do so.

A first necessity is, of course, a broad \emph{status quaestionis} yet in a
less traditional sense than one might expect: the focus here lies more upon
technical and technological concerns than specialised monographs for which data
was gathered manually. Nevertheless, I have for the sake of completeness chosen
to include a comprehensive bibliography of the field for clear reference. The
term bibliography would be rather less appropriate in describing the array of
databases and linguistic tools available to us - it is rather a select
catalogue documenting the most relevant items in the instrumentarium in some
detail while providing a briefly annotated list of other useful resources.

A second chapter is dedicated

\begin{enumerate}
\item a survey of previous studies on the language of the Greek papyri 
and an analysis of the methodology used therein;
\item a critical evaluation of the methods used by aforementioned 
studies, with equal attention for both positive and negative aspects of 
the applied method;
\item the development of a methodology based on this evaluation that is 
fit for use in corpus-based grammatical and linguistic studies;
\item an analysis and evaluation of the available tools in the field of 
papyrology for such studies, followed by suggestions for possible 
improvements to these tools;
\item the study of select grammatical and linguistic questions 
concerning the language of the Greek papyri using the methodology and 
toolset developed previously — essentially a practical test of the 
previous findings.
\end{enumerate}

