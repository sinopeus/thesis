%************************************************
\chapter{Preface}
\label{chap:preface}
\mtcaddchapter
%************************************************

Originally, this work was to be a treatise on select topics in the
linguistic study of the papyri; the development of natural language
processing tools for ancient Greek was only an accessory to the main
goal; but as I embarked upon studies in computer science, I
accordingly decided to apply newfound knowledge to the problem. The
tools became the main goal, and thus this work stands as it is.

I chose a recent approach to natural language processing which is
considerably complex in its implementation and set high hopes, which
have, in a sensen, been dashed ever so slightly. It was hard work, but
rewarding in its own right, and even though the experiment has not
been entirely successful, I hope to have demonstrated that there are
possibilities for further research in this grey zone between the
classics and computer science, which are sorely hampered by the lack
of resources, knowhow and manpower. The best I can do is lead by
example, even though I am but one man stretching to have one foot in
each camp, a decidedly unstable position if there ever was one.

I leave it up to others to decide whether my chosen pursuit is
worthwhile: as for myself, I know it was, and still will
be. \textit{Perferam et obdurabo.}
