
\chapter{Concepts and techniques}
\label{chap:conceptstechniques}
\section{Mathematics}
\label{sec:mathematics}


\subsection{Set theory}
\label{sec:settheory}

Consider an object $o$ and a set $A$. We write 'object $o$ is an
element of set $A$' as $o \in A$. Sets themselves are also objects and
can belong to other sets. Consider two sets, $A$ and $B$. We write
that $A$ is a subset $B$ as $A \subseteq B$; this implies that $B$
contains all elements in $A$ and does not exclude the possibility of
$A$ = $B$; if $A$ is a proper subset of $B$, i.e. all elements of $A$
are in $B$ but not all elements of $B$ are in $A$, we write $A \subset
B$. Mirroring these symbols from right to left gives us the symbols
for supersets and strict supersets, respectively. The empty set, which
contains no elements, is written as $\varnothing$.

There exist different binary operators on sets (i.e. operators on two
sets) which return another set. The most common operators are:
\begin{itemize}
\item the union of sets, written as $A \cap B$, denotes a set which
  contains all elements which are in $A$ and $B$;
\item the intersection of sets, written as $A \cup B$, denotes a set
  which contains all elements which are in both $A$ and $B$;
\item the difference of sets, written as $A \setminus B$, denotes a
  set which contains every element from $A$ excluding those which are
  also in $B$.
\item the Cartesian product of sets, written as $A \times B$, is the
  set containing all ordered pairs of elements from $A$ and $B$.
\end{itemize}

\subsection{Probability}
\label{sec:probability}

A \textbf{probability} is a measure for the likelihood of an event for an
experiment, which we intuitively understand to be an action whose
outcome we want to observe. Such events are then elements of a set
containing all possible outcomes of an experiment; an event can be a
point or subset of that set. We call this set the \textbf{sample space},
denoted S. We denote the probability of an event E as P(E).

Axiomatically, we can define probability as follows:

\begin{enumerate}
\item For every event $E$, $P(E) \geq 0$; no event can have a negative probability.
\item $P(S) = 1$; that is to say, every experiment has an event.
\item For any sequence of disjoint events $A_i$ (that is to say,
  there is no overlap between events), the probability of any one of
  these events occurring is the sum of their respective probabilities.
\end{enumerate}

A few other important properties of probabilities and events are the following:

\begin{enumerate}
\item The complement of an event $A$, which is the union of all
  elements of $S$ which are not an element of $A$, is denoted $A^C$;
  $P(A^C$ is ther probability of this event occurring and is equal to $1
  - P(A)$.
\item For any event E, $0 \leq P(E) \leq 1$.
\item Given any two events $A$ and $B$, $A \subset B \rightarrow P(A) \leq P(B)$.
\end{enumerate}

Given probabilities of a number of events, we can establish
relationships between these probabilities and compute related
probabilities using the rules certain rules. A classic rule is the
\textbf{multiplication rule}, which states that if we perform an
experiment in $k$ parts and the $i^{th}$ part of the experiment has
$n_i$ possible outcomes, and the outcomes of prior parts of the
experiment do not affect latter ones, the probability of any specific
sequence of partial outcome will be the product of all outcome counts
$n_i$ with $i$ ranging from $1$ to $k$.

Set theory is important when we want to know the probability of an
event E which can be constructed from a set of sets $A_i$ using set
operators. The third axiom of probability has already given us the
solution for disjoint events; events may also overlap, and in this
case, we need a more sophisticated formula. For the union of any $n$
events $A_i$, the following holds:
\begin{equation}
  P(\bigcup_{i=1}^n A_i) = \sum\limits_{i=1}^n P(A_i)
\end{equation}

For the intersection of these $n$ events $A_i$, it can be proven that:
\begin{equation}
  P(\bigcap_{i=1}^n A_i) = \lim_{n \to \infty} P(A_i)
\end{equation}
Knowing both these rules is important when considering
\textbf{conditional probability}; for two events $A$ and $B$, suppose
we know $B$ has occurred and we want to know what the probability of
$A$ occurring is given this information. We call this the conditional
probability of $A$ given $B$ and write it $P(A|B)$. If $P(B) > 0$,
then we define it as:

\begin{equation}
  P(A|B) = \frac{P(A \cap B)}{P(B)}
\end{equation}

Using this formula, we can directly derive \textbf{Bayes' theorem} as follows:

\begin{equation}
  \begin{aligned}
    P(A|B) &= \frac{P(A \cap B)}{P(B)} \\
    P(A|B) P(B) &= P(A \cap B)\\
    P(A|B) P(B) &= P(B \cap A)\\
    P(A|B) P(B) &= P(B|A) P(A)\\
    P(B|A) &= \frac{P(A|B) P(B)}{P(A)}
  \end{aligned}
\end{equation}

\subsection{Calculus and linear algebra}
\label{sec:calculus}
\begin{itemize}
\item Derivatives and computing extrema
\item Jacobian matrices
\item Numerical methods
\item Vector spaces
\end{itemize}

\subsection{Statistics}
\label{sec:statistics}
    \subsubsection{Regression}

    Regression is a classic technique from statistics, often visualised as
    'fitting a line to a set of points'. Classification is a related
    technique which uses regression to classify new data points. This is
    essentially what any probabilistic model for natural language
    processing does, in one form or another. What follows is an overview
    of various types of regression and corresponding methods for
    classification.

    We start with \textbf{univariate linear regression}. Given a set
    of $n$ points in the plane, we want to find a hypothesis that best
    corresponds to the location of these points, and we want this
    hypothesis to be a linear function. This function is then of the form:
    \begin{equation}
      h_w(x) = w_1x + w_0
    \end{equation}
    Unless all points are collinear, it is of course impossible to
    find a function of this form that gives a correct mapping for each
    point. The best we can do is find the values of $w_o$ and $w_1$ for
    which the empirical loss on the mappings is minimal. The traditional
    way of doing this is to define a function that computes the squares of
    the errors and sums it over all data points; this is called an
    \textbf{$L_2$ loss function}. We now want to find the values of $w_0$
    and $w_1$ for which this function attains a minimum. We can find these
    minimal points by solving for the roots of the partial derivative
    functions of this loss function with respect to $w_0$ and $w_1$.  This
    problem is mathematically relatively simple and has a unique
    solution. This solution is valid for all loss functions of this type.

    Problems arise when we are trying to create a nonlinear model. In
    this case, the minimum loss equations frequently do not have a unique
    solution. We can of course still model the problem algebraically, and
    the goal is the same: finding the roots of the partial derivative
    function. Now, however, we need to use a more sophisticated method:
    \textbf{gradient descent}. We can visualise this technique as
    'descending a hill'; the 'hill' is the graphical representation of the
    root of the system of partial derivatives, and by 'descending' this
    hill, i.e. by iteratively picking values which bring us closer to the
    bottom part of the valley next to the hill, which corresponds to the
    minimal point of the function, eventually convergence will be reached
    on the minimum and we will found the correct weights for our
    function. The difference by which we change the value at each
    iteration is called the \textbf{step} or \textbf{learning rate} and
    determines how fast we will converge; it may be either a fixed
    constant or a mutable value which can increase or decay according to
    the current state of our descent.

    \textbf{Multivariate linear regression} poses a similar problem;
    only this time the function is not dependent on a single variable, but
    on two or more. Such a function is a bit more complex, but we can
    find a solution to the regression problem using analogous
    techniques. Suppose the function has $n$ variables. Each example $x_j$
    must be a vector with $n$ values. At this point, we are looking at a
    function of the following form:
    \begin{equation}
      h_w(x_j) = w_0 + w_1x_{j,1} + w_2x_{j,2} + ... + w_nx_{j,n} = w_0 + \sum\limits_{i} w_ix_{j,i}
    \end{equation}
    We want to simplify this to make algebraic manipulations
    easier. We therefore prepend an extra component $x_{j,0} = 1$ to the
    vector $x_j$; now using vector notation we can simplify the
    previous equation to:
    \begin{equation}
      h_w(x_j) = \sum\limits_{i} w_ix_{j,i} = w \cdot x_j
    \end{equation}
    What we are now looking for is a vector $w$ containg the weights
    of our function which minimises the empirical loss, as in
    univariate linear regression. We can equivalently use gradient
    descent; only now, of course, the computational cost of that technique
    will be higher. A common problem can now appear: \textbf{overfitting},
    that is, giving an irrelevant dimension of the vector w too much
    weight due to chance errors in the computation. This can be
    compensated by taking into account the complexity of the hypothesis; a
    statistical equivalent to Ockham's razor, if you will.

    \subsubsection{Classification} 

    We can define an analogous process for classification; only now
    the function must not fit to the data itself but must create a
    \textbf{decision boundary} between data points. If there exists a
    linear function which satisfies this property for a given data set
    , we call the bounding line or surface generated by this function a
    \textbf{linear separator}, and the data set \textbf{linearly
      separable}. The hypothesis function is now of the form:
    \begin{equation}
      h_w(x) = 1$ if $ w \cdot x \geq 0$, $0$ otherwise.$
    \end{equation}
    We can view this as a function $threshold(w \cdot x)$ which is
    equal to 1 only if $w \cdot x = 0$ Note that while the separating
    function is linear, the hypothesis function is not, and in fact has
    the distinctly unappealing property of not being differentiable. We
    can therefore not apply the technique of gradient descent
    here. Furthermore, this type of function has exactly two outputs: 1 or
    0. For our purposes, we need subtler methods of classification.  This
    type of hypothesis function is therefore not fit for our purposes, but
    it does give a good idea of what classification is.

    The best option is replacing the hard threshold function with the
    sigmoid or logistic function, which offers a good approximation and is
    differentiable at every point. This function is of the form:
    \begin{equation}
      g(x) = \frac{1}{1 + e^{-x}}
    \end{equation}
    Such that our new hypothesis function is:
    \begin{equation}
      h_w(x) = g(w \cdot x) = \frac{1}{1 + e^{- w \cdot x}}
    \end{equation}
    If we use this function, we are performing \textbf{linear
      classification with logistic regression}.

\subsubsection{Logistic regression and the chain rule}
\subsubsection{Clustering}


\subsection{Formal language theory}
\label{sec:formalgrammars}
\begin{itemize}
\item Languages and strings
\item Regular languages
\item Context-free grammar and languages
\item The Chomsky hierarchy
\end{itemize}

\section{Natural language processing}
\label{sec:nlp}

\subsection{N-grams}
\label{sec:ngrams}

\subsection{Hidden Markov Models}
\label{sec:hmm-maxent}

\subsection{Viterbi decoding}
\label{sec:viterbi}

\section{Artificial intelligence and machine learning}
\label{sec:aiml}

\subsection{What is machine learning?}
\label{sec:statistics}
\begin{itemize}
\item Supervised learning
\item Unsupervised learning
\end{itemize}

\subsection{Neural networks}
\label{sec:neuralnetworks}

An artificial neural network is massively parallel distributed
processor made up of simple processing units, which has a natural
propensity for storing experiential knowledge and making it available
for use.

For the design of an architecture which allows us to solve the
problems above, we have taken our cues largely from recent work in
machine learning as applied to natural language processing. In
particular, we follow the approach set out in Weston \& Collobert 2008
and expanded in Weston et al. 2011, that is, the use of deep neural
networks for joint training on our chosen corpus. This section is
dedicated to a more expository overview of that architecture for the
mathematical layman.

An artificial neural network is a massively parallel processing system
constructed from simple interconnected processing units (called
neurons) which has the ability to learn by experience and store this
knowledge for later use. The term 'neural network' is due to the
resemblance of this architecture to the most powerful biological
processor known to exist, the human brain, which has a way of
functioning which is broadly analogous to this process. 

Artificial neural networks find their origin in a mathematical model
dating from before the first wave of artificial intelligence in 1956,
the McCulloch-Pitts Threshold Logical Unit, known also as the
McCulloch-Pitts neuron. Warren McCulloch was a psychiatrist and
neuroanatomist; Walter Pitts a mathematical prodigy. Both met at the
University of Chicago, where a neural modeling community led by the
mathematical physicist N. Rashevsky had been active in the years
preceding the publication in 1943 of the seminal paper A logical
calculus of the ideas immanent in nervous activity. 

Formally, we can define a neuron as a triplet (v,g,w) where:

\begin{itemize}
\item v is an input function which takes a number of inputs and computes their sum;
\item g is an activation function which is applied to the output of the input function and determines if the neuron 'fires';
\item w is an output function which receives its input from the activation
  function and distributes it over a number of outputs.
\end{itemize}

Given its structure, we can also see a neuron as a composite function
$F = w \circ g \circ v$. The combination of several of these units
using directed links forms a neural network. A link connecting a node
$i$ to a node $j$ transfers the output $a_i$ of node $i$ to node $j$
scaled by a numeric weight $w_{i,j}$ associated with that specific
link. This is the general model of a neural network; countless
variations on this theme have been developed for different purposes,
mainly by modifying the activation function and the interconnection of
neurons.

The activation function $g$ typically will be a hard threshold
function (an example of this is the original McCulloch-Pitts neuron),
which makes the neuron a perceptron, or a logistic (also known as a
sigmoid) function, in which case we term the neuron a sigmoid
perceptron. Both these functions are nonlinear; since each neuron
itself represents a composition of functions, the neuron itself is a
non-linear function; and since the entire network can also be seen as
a composite function (since it takes an input and gives an output) the
network can be viewed as a nonlinear function. Additionally, choosing
to use a logistic function as an activation function offers
mathematical possibilities, since it is differentiable. This offers
similar possibilities as the use of the logistic function for
regression (cf. \textit{supra}).

The links between nodes can be configured in different ways, which
each afford distinct advantages and disadvantages. Broadly, we can
distinguish two models. The simplest model is the \textbf{feed-forward
  network}, which can be represented as an acyclic directed graph. The
propagation of an input through this kind of network can be seen as a
stream, with posterior (downstream) nodes accepting outputs from prior
(upstream) nodes. This type of network is the most widely-used and is
used in the architecture. A more complex type is the \textbf{recurrent
  network}, which feeds its output back to its input and thus contains a
directed cycle; this type of network has interesting applications (for
example in handwriting recognition), as they resemble the neural
architecture of the brain more closely than feedforward networks do.

Feed-forward networks are often organised (to continue the stream
analogy) in a kind of waterfall structure using layers. The input is
the initial stream, the output is the final stream; in between, we may
place hidden layers, which are composed of neurons which take inputs
and outputs as any neuron does, but whose output is then immediately
transferred to a different neuron. Throughout the network, we can
equip the neurons in each layer with distinct activation functions and
link weights and in this way mold the learning process of the network
to our purpose.

Single-layer networks contain no hidden layers; the input is directly
connected to the output. Therefore, the output is a linear combination
of linear functions. This is undesirable in many cases. The main
problem, demonstrated early on in the development of neural network
theory, is the fact that such a network is unable to learn functions
that are not linearly separable; one such function is the XOR
function, which is a very simple logical operator. Despite this, such
neural networks are useful for many tasks, as they offer an efficient
way of performing logistic regression and linear classification.

Our interest lies in multi-layer networks, however. Multi-layer
networks contain one or more layer between the input and output layer,
which are called hidden layers. By cascading the input through all
these layers, we are in fact modeling a nonlinear function which
consists of nested nonlinear soft threshold functions as used in
logistic regression. The network can now be used to perform \textbf{nonlinear
  regression}. Different algorithms exist which can be used to train the
network; the most important one is the \textbf{backpropagation algorithm},
which is the equivalent of the loss reduction techniques used in
linear regression.

Suppose that a neural network models a vector-valued hypothesis
function $h_w$ which we want to fit to an example output vector
$y$. We can create a $L_2$ loss function E by taking the error on this
vector and squaring it. This function can be quite complex, but by
taking partial derivatives of this function, we can consider the
empirical loss on each output separately, like so:

\begin{equation}
  \begin{aligned} 
    \frac{\partial}{\partial w} E(w) &= \frac{\partial}{\partial w} \lvert y - h_w(x) \rvert^2 \\
    &= \frac{\partial}{\partial w} \sum (y_k - a_k(x))^2 \\
    &=  \sum \frac{\partial}{\partial w} (y_k - a_k(x))^2
  \end{aligned} 
\end{equation}

If the output function is to have $m$ outputs, instead of handling one
large problem, we can subdivide the problem into $m$ smaller
problems. This approach works if the network has no hidden layers, but
due to the fact that nothing is really known about the hidden layers
if we only look at the output layer, a new approach is necessary. This
is called backpropagation, a shorthand term for backward error
propagation.
\begin{itemize}
\item backpropagation
\end{itemize}

\subsection{Deep learning}
\label{sec:neuralnetworks}

% \section{Principles of statistical natural language processing} 
% \label{sub:principles-nlp}

% The crux of the proposed method is the use of the aforementioned TreeTagger,
% developed by Helmut Schmid, which can lemmatize and tag at the same time,
% followed by another iteration using the Stanford Parser to analyse sentence
% dependencies. TreeTagger can be trained for any language, and its efficiency
% for highly inflected languages such as ancient Greek is due to its combination
% of two tagging methods, namely n-gram tagging and binary decision tree tagging,
% the combination of which makes TreeTagger the fastest part-of-speech tagger
% around. The Stanford Parser is the state of the art in sentence parsing.

% The following section, therefore, is intended to provide some theoretical
% background on statistical natural language processing. I have based myself upon
% Manning and Schütze's \textit{Foundations of Statistical Natural Language
% Processing} \citep{manning1999}, the current standard reference work, as well
% as \citet{koshy2004} and \citet{hopcroft2001} for concepts relating to automata
% theory, as well as on \citet{bod2004} for the paragraphs on statistics.
% Furthermore, to illuminate some aspects of TreeTagger's workings, I have also
% integrated Schmid's articles on tagging methodology as he applied it in
% TreeTagger [\citeyear{schmid1994} and \citeyear{schmid1995}]. 

% \subsection{Context-free grammar and pushdown automata} % (fold)
% \label{sub:formallang}

% The basis of all statistical natural language processing is the
% conceptualisation of natural language within the Chomsky hierarchy as a formal
% language generated by a \textbf{stochastic context-free grammar}, that is to say, a
% context-free grammar whose production rules are augmented by a probability. In
% computer science, such a language is termed a context-free language and defined
% as any language that can be recognized by a \textbf{nondeterministic pushdown
% automaton}. A nondeterministic pushdown automaton is understood to be a
% nondeterministic finite state automaton with access to an infinite stack, a
% finite state automaton being an automaton which is in a state that can
% transition into another state when triggered by input. Put more simply, a
% nondeterministic pushdown automaton possesses the following:

% \begin{itemize} 

% \item a \textbf{stack}, that is, a string of symbols which functions as the
%   automaton's memory. The automaton only has access to the leftmost symbol in
%   this string, a symbol which is dubbed the top of the stack; the top of the
%   stack can be used to determine which transition the automaton will make
%   next;

% \item an \textbf{input tape} whose symbols are scanned one by one just like a
%   finite-state automaton; 

% \item a \textbf{finite state control unit}, which controls the state of the
%   automaton. Normally this is done in response to input, but it is possible
%   in some types of automata to transition into a new state without any
%   input, as is the case for pushdown automata; this type of transition is
%   termed $\epsilon$-transition or $\lambda$-transition.

% \end{itemize}

% A simple example of finite state automaton is a coin-operated turnstile: it has
% two possible states, locked and unlocked, and two possible inputs, the
% insertion of a coin or a push. Depending on the state the machine is currently
% in, each of these inputs will trigger either a change or leave the machine in
% its current state. For instance, if the turnstile is unlocked, a push will make
% it turn and lock again; inserting a coin will do nothing.  Conversely, when it
% is locked, a coin will unlock it, while a push will still leave it locked. This
% type of automaton is deterministic, that is, there is only one next possible
% state to transition to. A nondeterministic finite state automaton would for
% example be a vending machine, which functions in a similar way but allows for
% transitions to various states depending on input; different combinations of
% coins and button presses will unlock different mechanisms in the machine and
% let a specific item fall into the bottom compartment.

% \begin{figure}
%   \begin{tikzpicture}[shorten >=1pt,node distance=5cm,on grid,auto] 
%     \node[text width=2cm, text centered, state] (locked)   {$locked$}; 
%     \node[text width=2cm, text centered, state](unlocked) [right=of locked] {$unlocked$};
%     \node[text centered, node distance=2cm] (coin) [above left=of locked] {$coin$};
%     \path[->] 
%     (coin) edge [] node {} (locked)
%     (locked) edge  [loop left] node {push} ()
%     edge  [bend left=20] node {coin} (unlocked)
%     (unlocked) edge  [bend left=20] node {push} (locked)
%     edge [loop right] node {coin} ();
%   \end{tikzpicture}
%   \caption{Transition diagram for a deterministic finite state automaton modeling a turnstile. Adapted from \citet[762]{koshy2004}.} \label{fig:turnstile}
% \end{figure}

% When a stack is added to this setup, the automaton is essentially provided with
% an infinite memory. Transitions will now be based upon the current state, the
% input symbol and the symbol at the top of the stack. An $\epsilon$-transition
% is possible as well, with $\epsilon$ replacing the input symbol. Thus, any
% transition now consists of the following: the consumption of input, the
% transition to a new state itself, and the replacement of the top of the stack
% by any symbol (it is even possible for the current symbol to remain in place).

% Natural language fits within this paradigm. Given an alphabet $\Sigma$ = \{a,
% b, c, d, \ldots\} or even, in the case of Greek, \{$\alpha$, $\beta$, $\gamma$,
% $\delta$, \ldots\} the initial state and top of the stack will give rise to a
% sequence of characters or strings formed from the alphabet, which may be a
% word, a sentence, an paragraph, \ldots In other words, context-free grammar
% provides a simple yet precise method for mathematically describing and studying
% the rules which govern the construction of natural language from smaller
% blocks; one can parse generated strings (in themselves context-free languages)
% and by induction assemble a grammar. It is also possible to feed a string as
% input to a pushdown automaton applying a context-free grammar; this will show
% whether the string is acceptable by the rules of the grammar or not.

% \begin{figure}
%   \begin{center}
%     \begin{tikzpicture}[shorten >=1pt,node distance=3cm,on grid,auto, label distance=0.4cm] 
%       \node[text centered] (input)   {$input$}; 
%       \node[text centered, align=center, state, rectangle, minimum size=2cm] (fscontrol) [right=of input] {finite\\state\\control};
%       \node[text centered] (acceptreject) [right=of fscontrol] {$accept/reject$};
%       \node[text centered,rectangle split, rectangle split parts=3, draw, label=0:$stack$, minimum size=1.2cm] (stack) [below=of fscontrol] {$top$};
%       \path[->] 
%       (input) edge [] node {} (fscontrol)
%       (fscontrol) edge  [] node {} (acceptreject)
%       edge  [bend left=20] node {} (stack)
%       (stack) edge  [bend left=20] node {} (fscontrol);
%     \end{tikzpicture}
%   \end{center}
%   \caption{General structure of a pushdown automaton. Adapted from \citet[220]{hopcroft2001}.} \label{fig:pushdownautomaton}
% \end{figure}