%*******************************************************
% Abstract+Sommario
%*******************************************************

\begingroup
\let\clearpage\relax
\let\cleardoublepage\relax
\let\cleardoublepage\relax

% \chapter*{Abstract}

% \vfill

\selectlanguage{dutch}
\pdfbookmark[1]{Samenvatting}{Samenvatting}
\chapter*{Samenvatting}

%motivation
Ondanks de bewezen diensten van corpusgebaseerde methoden in het
taalkundig onderzoek is er een groot gebrek aan geannoteerde digitale
corpora voor het Oudgrieks. Grote hoeveelheden tekst zijn ondertussen
digitaal beschikbaar, maar door gebrek aan mankracht is het niet
mogelijk deze handmatig te analyseren. Een betrouwbare methode om dit
probleem automatisch aan te pakken, al zou deze niet perfect zijn, zou
een stap in de goede richting zijn. Een belangrijk pijnpunt is dat dit
gebrek aan geannoteerde corpora het ook zeer moeilijk maakt om dit
soort methode te ontwikkelen.

% problem statement & approach
In dit werk trachten we dit pijnpunt te omzeilen door een dergelijke
methode te ontwikkelen met behulp van spitstechnieken uit de
artificiële intelligentie en bij wijze van \textit{case study} toe te
passen op een corpus gedigitaliseerde documentaire papyri. Deze
methode berust op een implementatie van een zgn. artificieel neuraal
netwerk, d.i. een wiskundig model dat in staat is om patronen te
herkennen en te leren. We passen de methode beschreven in
\cite{collobert-2011} toe op het Oudgrieks.

Het leerproces van dit netwerk wordt opgedeeld in twee fasen. In een
eerste wordt gebruik gemaakt van een groot ongeannoteerd corpus om via
een eenvoudig criterium een wiskundig model op te bouwen dat woorden
kan kaderen binnen het taalgebruik in het geheel.  Dit model kent
kansen toe aan reeksen woorden al naargelang die al dan niet 'correct'
zijn door te schatten of gelijkaardige reeksen kunnen voorkomen in het
corpus.

Een tweede fase schakelt over op een kleiner, geannoteerd corpus. Na
het defini\"eren van een wiskundige voorstelling voor morfologische en
syntactische categorie\"en worden meerdere netwerken ge\"initialiseerd
die gebruik maken van de voorheen opgebouwde wiskundige
representatie. De modellen worden afgestemd op de woord-annotatieparen
en be\"invloeden elkaar onderling om zo goed mogelijk gebruik te maken
van patronen die van potentieel nut zijn voor de taak van elk netwerk.

We waren van plan om het zo bekomen model toe te passen op een
digitaal beschikbaar corpus papyri, en dit na afloop in een
verdeelbaar formaat om te zetten en te opensourcen voor verdere
verwerking. De finale resultaten waren echter teleurstellend: het
systeem is te eenvoudig om correcte morfologische en syntactische
inferenties te maken en lijdt nog steeds onder de kleine hoeveelheden
geannoteerde tekst die beschikbaar zijn voor Oudgrieks. De evolutie
van het model dat ontwikkeld werd in de eerste fase van het leerproces
is echter positief en stemt hoopvol. Met langere rekentijd kan dit
type model zeker van nut zijn in toekomstig onderzoek. We stellen ook
een aantal pistes voor om ook de tweede fase van het leerproces te
verbeteren.






% Problem statement:
% What problem are you trying to solve? What is the scope of your work (a generalized approach, or for a specific situation)? Be careful not to use too much jargon. In some cases it is appropriate to put the problem statement before the motivation, but usually this only works if most readers already understand why the problem is important.
% Motivation:
% Why do we care about the problem and the results? If the problem isn't obviously "interesting" it might be better to put motivation first; but if your work is incremental progress on a problem that is widely recognized as important, then it is probably better to put the problem statement first to indicate which piece of the larger problem you are breaking off to work on. This section should include the importance of your work, the difficulty of the area, and the impact it might have if successful.
% Approach:
% How did you go about solving or making progress on the problem? Did you use simulation, analytic models, prototype construction, or analysis of field data for an actual product? What was the extent of your work (did you look at one application program or a hundred programs in twenty different programming languages?) What important variables did you control, ignore, or measure?
% Results:
% What's the answer? Specifically, most good computer architecture papers conclude that something is so many percent faster, cheaper, smaller, or otherwise better than something else. Put the result there, in numbers. Avoid vague, hand-waving results such as "very", "small", or "significant." If you must be vague, you are only given license to do so when you can talk about orders-of-magnitude improvement. There is a tension here in that you should not provide numbers that can be easily misinterpreted, but on the other hand you don't have room for all the caveats.
% Conclusions:
% What are the implications of your answer? Is it going to change the world (unlikely), be a significant "win", be a nice hack, or simply serve as a road sign indicating that this path is a waste of time (all of the previous results are useful). Are your results general, potentially generalizable, or specific to a particular case?

\selectlanguage{british}

\endgroup			

\vfill
Deze masterproef bevat \textbf{59840
} tekens.

