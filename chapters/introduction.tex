%************************************************
\chapter{Thesis} 
\label{chp:introduction}
%\minitoc\mtcskip
%************************************************
\section{Statement}

The following work deals with the application of techniques from the
field of machine learning and natural language processing to ancient
Greek. We intend to show that it is possible to use these techniques
to generate linguistic annotation in an accurate and efficient manner.

In order to achieve this, we have implemented a state-of-the-art machine
learning architecture for natural language processing, proposed in
\cite{collobert2008,collobert-2011}. The chosen approach makes use of
raw Greek text to counterbalance the relative scarcity of annotated
ancient Greek corpora, which are essential in most natural language
processing systems.

We evaluate our method by measuring the absolute performance of the
model against a validation corpus and executing a case study on a
freely available digital corpus of documentary papyri.

\section{Motivation}
We wish to demonstrate the potential of a methodology based on
computational techniques for the study of the classical languages. The
field has seen a move towards digitalisation in the past half century,
but a lot of potential is left untapped. Steps in the right direction
are currently being made, but we can drive progress much further. The
classicist breed does not number many specimens; and though the true
classics, the Homers, the Platos, the Virgils, have been subjected to
thorough analysis for millennia, the amount of texts in need of
scholarly attention remains large. Demonstrating the potential of
computational methods for Greek linguistics will hopefully serve as
further proof of their potential for other branches of classics, such
as stylometry, authorship verification, textual criticism, and more.

Specifically, we want to apply such techniques to the problem of
linguistic annotation in order to open new perspectives on the
structure and evolution of the Greek language. Massive amounts of
source material are now digitally available: resources such as the
\textit{Thesaurus Linguae Graecae} and the Perseus project provide us
with dozens of millions of words. What has been lacking is the
systematic development of large annotated corpora. While systems have
been put in place which offer morphological and lemmatic analysis in a
rudimentary form, this cannot serve as a full linguistic corpus; of
these, only two of note exist, which together number about half a
million words and thus are relatively small compared to the vast
textual repository of the TLG. We need to expand this type of resource
to many more texts. The problem is that using manual methods to tag
the large Greek corpora is rather prohibitive due to their size. By
making use of computational methods, we can make an attempt at
offering a relatively complete linguistic corpus for ancient Greek.

The chosen case study, the annotation of a corpus of documentary
papyri, is meant to evoke the possibilities afforded by these methods
by directly applying them to a corpus which has not been subjected to
a modern linguistic study. The question of the language of the papyri
has in the past thirty years seen little evolution until the recent
appearance of \cite{lpapyri}, which has placed the subject in the
spotlight again. Twentieth-century scholarship on the topic, though
still useful for those interested in the study of the papyri for
historical purposes, is either antiquated, limited in scope or
incomplete; for more on this, see section \vref{sec:lpapyri}. 

Despite this, the papyri are useful source material for the history
and evolution of the Greek language: the corpus consists of more than
4.5 million words, spanning more than a thousand years and many
different discourse registers. Annotating this corpus would be a boon
to scholars interested in the Greek of the papyri and Greek historical
linguistics in general, as it would facilitate the creation of
linguistically sound grammars and lexica.

\section{Outline}
We begin by giving an overview of the background for this thesis.
First the historical and linguistic background of the question is
handled in chapter \vref{chp:background}. We give a historical overview of
previous efforts to study the grammar of the papyri, as well as of the
applications of the techniques of corpus linguistics to the Greek
language in general.

Secondly, in section \vref{sec:shortconcepts}, we illustrate critical
concepts and techniques for the task at hand which are necessary to
understand the underpinnings of the applied methodology. Formalism is
kept to a minimum; for that, we refer to appendix
\vref{chp:conceptstechniques}.

We proceed with an overview of directly related work in
section \vref{sec:relatedwork}. A problem statement and proposed solution is
offered in chapter \vref{chp:analysis} with specific reference to the
scholarly context; jointly, we extend a brief overview of the
possibilities the proposed method offers.

The theory behind the natural language processing architecture, as
well as a method to represent Greek morphological annotations
mathematically are illustrated in chapter \vref{chp:design}. The
implementation is detailed in chapter \vref{chp:implementation}: we provide
details on the choice of programming language, the availability of the
source code, the technical requirements for running our code, etc. The
source code is found in appendix \vref{chp:sourcecode}.

We detail our results in chapter \vref{chp:results} with a critical
assessment in chapter \vref{chp:assessment}, where we state our
contribution. Section \vref{sec:further_work} of this chapter is
specifically dedicated to an overview of future avenues for research
on this type of method. Finally, we summarize our findings in chapter
\vref{sec:conclusion}.

