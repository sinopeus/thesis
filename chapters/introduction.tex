%************************************************
\chapter{Introduction} 
\label{chp:introduction}
%\minitoc\mtcskip
%************************************************

The study of the language of the papyri has in the past thirty years
seen little evolution until the recent appearance of Evans and Obbink's
\textit{The Language of the Papyri} \citep{lpapyri}, which has placed
the subject in the spotlight again. Twentieth-century scholarship on the
topic, though still useful for those interested in the study of the
papyri for historical purposes, is either antiquated, limited in scope
or incomplete (see \textit{infra}). Despite this, the papyri are
useful source material for the history and evolution of the Greek
language, as they contain not only official texts but private
documents as well, whose linguistic features and peculiarities have the
potential to foster new insights into the nature of colloquial Greek.

\section{Thesis}
The following work intends to prove that it is possible to generate
basic linguistic annotation for a large digitalised corpus of papyri
in ancient Greek using techniques from the field of natural language
processing.  Such a corpus could be a boon to scholars interested in
the Greek of the papyri, as it would facilitate, for instance, the
creation of linguistically sound grammars and lexica.

The basic idea is the one proposed in two articles by H. Dik and R.
Whaling, [\citeauthor{dik2008}, \citeyear{dik2008} and
\citeyear{dik2009}], in which they document their method for
semi-automatically tagging the Perseus project's texts under their own
framework, PhiloLogic. They start with a database of analysed forms
and a series of tagged texts which they use as initial data to train a
decision tree tagger, TreeTagger, developed by Helmut Schmid at the
University of Stuttgart, a tool which despite being developed in 1995
has aged well as far as performance is concerned. They achieved
remarkable accuracy: with refinements to the training data they
achieved 96.2\% accuracy during tests on the original training data
and 91\% accuracy on new data, a result which compares quite favorably
when compared to TreeTagger's 97\% accuracy when used on German
newspaper articles considering the high complexity of ancient Greek
and the variety of styles of ancient Greek literature.

It occurred to me that this might be a great method for processing the
corpus of papyri with a relatively small effort for a high payoff;
using data from the Perseus and PROIEL projects, it could be possible
to train TreeTagger for both morphology and syntax, apply the
resulting parameters to the corpus and thus for the greatest part
obviate the need for manual tagging.  Given the extent of the corpus
(about 50,000 texts containg almost 4,500,000 words), achieving even
85\% accuracy would reduce the amount of untagged words to 675,000,
many of which I would expect to be proper names or morphologically
`erroneous' forms as are often found in the papyri, data which could
itself undergo additional processing, though that is a task not easily
automated en rather beyond the scope of this work.  \footnote{As I set
out to verify the originality of my thesis, I found that this
statistical approach has been used before for textual criticism!
\textit{Vide} \citet{mimno2009}, an abstract of which may be found at
\url{http://people.cs.umass.edu/~wallach/publications/mimno09computational.txt}.}
\section{Structure}

We begin by giving an overview of the background for this thesis:
first historical and linguistic, expounding on previous efforts to
study the grammar and the papyri and to apply the techniques of corpus
linguistics to the Greek language and general, second technical,
describing a set of core concepts and techniques which are relevant to
the task at hand.

This is followed by an analysis of the objectives and requirements, as
well as a general (that is, only described in broad strokes, without
providing full details of the algorithms and implementation)
methodology. Correspondingly, a chapter is dedicated to the general
algorithmic design and structure of our program, followed by a chapter
delving into the actual implementation, which provides more details on
the choice of programming language, the availability of the source
code, the technical requirements for running our code, etc.

We then offer an interpretation and a critical assessment of our
program's output. Special attention is, of course, given to the
linguistic accuracy of our results. A final chapter is dedicated to an
overview of further possibilities for research in this field and of
possible applications outside papyrology and to the field of Greek
linguistics in general.

\section{Contributions}
% \section{Methodology}

