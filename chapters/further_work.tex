% **************************************
\chapter{Further work} % (fold)
\label{chp:further_work} 
% **************************************


\section{Improving the language model}
\subsection{Larger training sets}
\begin{itemize}
\item mainly: TLG holds much more text but does not free it
\item connection with projects like Open Philology: build large tagged
corpora using crowdsourcing
\end{itemize}
\subsection{Linguistic foreknowledge}
\begin{itemize}
\item simplicity is strength of model
\item can we use foreknowledge strategically?
\end{itemize}
\subsection{Integration with other resources}
\subsection{Expanding the range of tasks}
\subsubsection{Stemming}
\subsubsection{Named entity recognition}
\subsubsection{Semantic role labeling}
\subsubsection{Deep parsing}
% \section{Named entity recognition}
% \label{sec:ner}

% Named entity recognition is a subdiscipline in natural language
% processing which is concerned with the automatic extraction and
% localisation of all kinds of names from texts. It has been used
% extensively in literary texts with a view to discern the importance of
% certain characters throughout the text. The KU Leuven's long-standing
% Prosopographia Ptolemaica project, which aims to be a repository of
% all inhabitants of Egypt between 300 and 30 B.C., could easily benefit
% from these techniques. The abundant manual labour that has gone into
% the project could be fed as training data to and then supplemented by
% a named entity recognition engine that could also categorise personal
% names by any criteria and establish contextual relations between
% them. To take a very rudimentary example, the name 'Alexander' could
% be retrieved in all texts and a cluster of related names generated, so
% that related individuals may be placed in a web of relations; or one
% could ask, by combining the already present linguistic annotation, to
% display all adjectives which accompany the name 'Alexander'.

% It could even go further than this and also include other particular
% names, such as places, distances, monetary units, weights, and so
% on. Historians could create a comprehensive overview of, for instance,
% the inflation of Egyptian currency, or map out trade connections using
% a search for all mentions of currency, weight and places which are in
% proximity to each other.

\section{Refining the corpus}
\label{sec:refiningcorpus}

\subsection{Collaborative editing}
\label{sec:collaborative}
% The IDP project is heading into crowdsourcing territory at full steam,
% and excluding our own work from this movement would be inserting a
% shrill note into this symphony. All data is placed on GitHub at
% \url{https://github.com/sinopeus/tjufy}, freely accessible and
% editable for all.

% This opens promising avenues of inquiry that do not have a direct
% relation to this thesis. For instance, it can in the future be
% integrated into SoSOL and absorbed into the larger codebase for the
% IDP project if it does not create too much overhead for the current
% developers (the technical back end of the IDP seems to be labyrinthine
% and adding new layers of complexity might be off-putting). It could
% even grow into a separate project which itself could be linked to
% \texttt{papyri.info} as the HGV, APIS and Trismegistos currently are,
% by a common system of indexation.

% Making the code publically available to all also has the advantage of
% the public eye inspecting the texts; using solely automatic analysis
% is bound to deliver an inaccurate result, however small that
% inaccuracy may be, as creating a NLP engine perfectly capable of
% understanding language would be the equivalent of creating a perfect
% artificial intelligence. Therefore, considering the size of the
% corpus, one must rely upon the intelligence of the community. In the
% same way open source software backed by large communities is often of
% excellent quality due to public inspection, the potential for a
% crowdsourced corpus is immense.

\subsection{Integration with other resources}
\label{sec:linking}

\subsubsection{D. Bamman}
\cite{bammanpbml2008,bammantlt8,bammancrane2011}

\subsubsection{The Open Philology Project}
\begin{itemize}
\item possible to integrate papyri?
\item also uses automated approaches
\end{itemize}

