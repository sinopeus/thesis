%************************************************
\chapter{The language of the papyri}
\label{chp:printedworks}
%\minitoc\mtcskip
% \mtcaddchapter
%************************************************
The earliest compendious grammar of the papyri limits itself to the Ptolemaic
era but explores it at great length. The work consists of a part on phonology
and morphology, made up of three slimmer volumes, and a part on syntax,
encompassing three larger volumes. Its composition seems to have been
exhausting: it took Mayser thirty-six years to finish volumes I.2 through II.3,
with I.1 only completed in 1970 by Hans Schmoll, at which point the entire
series was given a second edition.

When casually browsing through some of its chapters (though casual is hardly
the word one would associate with the \text{Grammatik}) it is remarkable to see
that Mayser brings an abundance of material to the table for each grammatical
observation he makes, however small it may be. For instance, the section on
diminutives essentially consists of pages upon pages of examples categorised by
their endings.

This is its great strength as a reference work - whenever one is faced with an
unusual grammatical phenomenon in any papyrus, consulting Mayser is bound to
clarify the matter; or rather, it was, for the work is now inevitably dated.
The volumes published during Mayser's lifetime only include papyri up to their
date of publication; only the first tome by Schmoll includes papyri up to 1968.
It is still a largely useful resource, but it is in urgent need of refreshment.

After Mayser set the standard for the Ptolemaic papyri, a grammar of the
post-Ptolemaic papyri was a new \textit{desideratum} in papyrology. The work
had been embarked on by Salonius, Ljungvik, Kapsomenos, and Palmer, only to be
interrupted or thwarted by circumstance or lack of resources.
\citet{salonius1927}, for instance, only managed to write an introduction on
the sources, though he offered valuable comments on the matter of deciding how
close to spoken language a piece of writing is. \citet{ljungvik1932} contains
select studies on some points of syntax.

It is in the 1930's that we see attempts to create a grammar of the papyri that
would be the equivalent of Mayser for the post-Ptolemaic period.
\citeauthor{kapsomenos1938} published a series of critical notes
[\citeyear{kapsomenos1938}, \citeyear{kapsomenos1957}] on the
subject; though he attempted at a work on the scale of the \textit{Grammatik},
he found the resources sorely lacking, as the existing editions of papyrus
texts could not form the basis for a systematic grammatical study. The other
was \citeauthor{palmer1934}, who had embarked on similar project and had
already set out a methodology [\citeyear{palmer1934}]; the war interrupted his
efforts, and he published what he had already completed, a treatise on the
suffixes in word formation [\citeyear{palmer1945}].

A new work of some magnitude presents itself two decades later with B. G.
Mandilaras' \textit{The verb in the Greek non-literary papyri}
[\citeyear{mandilaras1973}]. Though it does not aim to be a grammar of the
papyri, it does offer a thorough and satisfactory treatment of its namesake.
Further efforts essentially do not appear until the publication of Gignac's
grammar. It is essentially treading in the footsteps of Mayser, only with
further methodological refinement and a more limited, though still sufficiently
exhaustive, array of examples. The author, for reasons unknown to me, only
managed to complete two of the three projected volumes, on phonology and on
morphology. The volume on syntax is thus absent, a gap only partly filled by
Mandilaras' \textit{The verb in the Greek non-literary papyri}.

\begin{itemize}
\item focused on a specific area
\item supersedes Mayser - supplements Gignac in a way
\end{itemize}

\begin{itemize}
\item the newest work - a collection
\item somewhat more of an appetiser
\item still offers some interesting articles
\item carves a path for new research
\end{itemize}
