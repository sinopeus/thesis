%************************************************
\chapter{Digital resources}
\label{chp:digitalresources}
\raggedbottom 
\minitoc\mtcskip
%\flussbottom
%************************************************
%\section{The digital classics: a brief history}
%\label{sect:corporaclassics}

\subsection{Format}
\label{subsect:format}
Raw information is useless; in order to harness it in a meaningful way, one
must mold it into a usable shape, or at least create a prototypical shape into
which the information can be molded. While this is true for any kind of
information, when computers are involved, it is doubly so. Modern-day computers
are capable of processing enormous amounts of data at incredible speeds; yet
they require all data to be provided in a strictly structured and regular
manner that they can analyse.

 Corpora cannot escape this rule; and the fact
that they are samples of natural language, a system that is structured but
diverse and often ambiguous, makes their conversion into a form readable by
computers a complex task. Different methods have been devised to achieve this;
over the years, some have been fallen into obsolescence, while others have
evolved and coalesced into a few standards that are now in general use.
\subsubsection{Encoding}
\label{subsubsect:encoding}
Computers use code to transmit or store information, and thus also transmit or
store numbers and text; the format of such code is termed encoding. Various
such encoding systems exist for all kinds of purposes. I have limited myself to
the most relevant for the encoding of texts in ancient Greek; a few other
legacy encodings do exist but are rarely used.

 \textsc{Beta Code} was
developed by David Packard in the late 1970's. It is not strictly an encoding,
but rather a method of representing polytonic Greek using solely ASCII
characters. The format was adopted by the TLG in 1980 and has spread in use
since; its main attraction is its adherence to the ASCII encoding, which allows
the user to type Greek without switching keyboard layouts or resorting to
different encodings. Beta Code may easily be converted to other encodings, most
importantly Unicode.

 \textsc{Unicode} is not solely restricted to Greek, but
aims to provide an encoding that can represent and treat text in a wide variety
of writing systems. The great benefit of this encoding is that it effectively
unites a variety of encodings and thus eliminates incompatibilities. The main
problem, then, is incompatibility with Unicode itself, but as its success
grows, the number of systems and applications not supporting Unicode is ever
decreasing; recently developed or updated software is highly likely to be
compatible with Unicode. As far as Greek is concerned, Unicode is also carving
out a place for itself to the detriment of Beta Code; using Unicode ensures
correct rendering of characters in all possible environments without requiring
specialised fonts to be installed, and the conversion from Beta Code can be
omitted. It is for this reason that the IDP project has recently converted all
of its Greek text from Beta Code to Unicode. Others may yet follow.


\subsubsection{Markup}
\label{subsect:markup}
\begin{itemize}
\item introduction
\item SGML
\item XML
\item TEI
\item EpiDoc
\end{itemize}
\subsection{Annotation}
\label{subsect:annotation}
\subsubsection{Morphological annotation}
\subsubsection{Morphosyntactic annotation}
\subsubsection{Syntactic annotation}
\subsubsection{Semantic annotation}
\subsubsection{Discourse annotation}
\subsection{Distribution}
\label{subsect:distribution}
The methods of distribution for corpora have, just like all other information,
dramatically changed in the course of the last fifty years. Before the advent
of computing, a scholar wishing to consult a concordance had to have access to
a library, where he could peruse the hefty tomes; the first computerised
corpora were to be consulted on mainframes that often were nearly the size of
the room they occupied. The advent of mobile storage, such as diskettes, CDs
and DVDs increased portability; yet nothing has improved accessibility to the
same degree as the Internet has.

 Distributing a corpus via Internet can be
done in a few ways. A first group are read-only: their databases are only
accessible through a dedicated interface, usually a website or a special server
that must be accessed through a specific program or transfer protocol. The
Library of Latin Texts, for instance, has to be accessed through Brepols'
dedicated website. 

 A second group allows the downloading of raw data, to be
used as the user wishes. The Perseus Project allows this; IDP has gone farther
and has made all its data available on GitHub, a widely-used service for
hosting software development projects. GitHub allows coordination of
development and provides a constant stream of updates that can easily be
downloaded.

 It is this group that is of greater interest, as raw data suits
itself well to manipulation and subsequent analysis. The GitHub page for the
IDP project may be found at \url{https://github.com/papyri}. Further
indications as to how to acquire and use this data are provided in the part on
methodology.
\subsection{Extant corpora}
\label{subsect:extantcorpora}
\subsubsection{Historical corpora}
\label{subsubsect:historicalcorpora}
\subsubsection{Dialectal corpora}
\label{subsubsect:dialectalcorpora}
\section{Tools}
\label{sect:tools}
\subsection{Lexica}
\label{subsect:lexica}
\begin{itemize}
\item Liddell-Scott-Jones
\item Intermediate LSJ
\item Slater's Lexicon to Pindar
\item Bailly
\item Autenrieth
\item New Testament Greek Lexicon
\item Autenrieth
\end{itemize}
\subsection{Grammars}
\label{subsect:grammars}
\begin{itemize}
\item Herbert Weir Smyth
\item K\"{u}hner - Gerth
\item Gildersleeve - Syntax
\item Goodwin - Syntax
\item De Witt Burton - Syntax of the Moods and Tenses in New Testament Greek
\item A Grammar of Septuagint Greek
\end{itemize}
\subsection{Text browsers}
\label{subsect:textbrowsers}
\begin{itemize}
\item Perseus Java Hopper
\item Diogenes
\item Andromeda
\item PhiloLogic
\item Musaios
\item A Grammar of Septuagint Greek
\end{itemize}
\subsection{Search tools}
\label{subsect:searchtools}
\subsection{Morphological parsers}
\label{subsect:morphparsers}
\subsection{Part-of-speech tagging}
\label{subsect:postagging}
\section{Bibliography}
\label{sect:digitalbibliography}
