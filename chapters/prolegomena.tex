%************************************************
\chapter{Introduction}
\label{chp:introduction}
\minitoc\mtcskip
%************************************************


A\footnote{The following section is based \emph{passim} on
\citet{okeeffe2010}.} corpus or text corpus is a large, structured collection
of texts designed for the statistical testing of linguistic hypotheses. The
core methodological concepts of this mode of analysis may be found in the
concordance, a tool first created by biblical scholars in the Middle Ages as an
aid in exegesis. Among literary scholars, the concordance also enjoyed use,
although to a lesser degree; the eighteenth century saw the creation of a
concordance to Shakespeare.

 The development of the concordance into the modern corpus was not primarily
 driven by the methods of biblical and literary scholars; rather, lexicography
 and pre-Chomskyan structural linguistics played a crucial role.

 Samuel Johnson created his famous comprehensive dictionary of English by means
 of a manually composed corpus consisting of countless slips of paper detailing
 contemporary usage. A similar method was used in the 1880s for the Oxford
 English Dictionary project - a staggering three million slips formed the basis
 from which the dictionary was compiled.

 1950s American structuralist linguistics was the other prong of progress; its
 heralding of linguistic data as a central given in the study of language
 supported by the ancient method of searching and indexing ensures its
 proponents may be called the forerunners of corpus linguistics.

Computer-generated concordances make their appearance in the late 1950s,
initially relying on the clunky tools of the day - punch cards. A notable
example is the \emph{Index Thomisticus\index{Index Thomisticus}}, a concordance
to the works of Thomas of Aquino created by the late Roberto Busa S.J. which
only saw completion after thirty years of hard work; the printed version spans
56 volumes and is a testament to the diligence and industry of its author. The
1970s brought strides forward in technology, with the creation of computerised
systems to replace catalogue indexing cards, a change that greatly benefited
bibliography and archivistics.

 It is only in the 1980s and 1990s that are marked the arrival of fully
 developed corpora in the modern sense of the word; for though the basic
 concepts of corpus linguistics were already widely used, they could not be
 applied on a large scale without the adequate tools. The rise of the desktop
 computer and the Internet as well as the seemingly ever-rising pace of
 technological development ensured the accessibility of digital tools.  The old
 tools - punch cards, mainframes, tape recorders and the like - were gladly
 cast aside in favour of the new data carriers.

 The perpetual increase of computing power equally demonstrated the limits of
 large-scale corpora; while lexicographical projects that had as their purpose
 to document the greatest number of possible usages could keep increasing the
 size of their corpora, the size of others went down as they whittled the data
 down to a specific set of uses of language.

 The possible applications of the techniques of corpus linguistics are diverse
 and numerous; for they allow for a radical enlargement in scope while
 remaining empirical, and remove arduous manual labour from the equation.
 Corpus linguistics can be an end to itself; it can, however, assert an
 important role in broader research.  \citet[7]{okeeffe2010} mention areas such
 language teaching and learning, discourse analysis, literary stylistics,
 forensic linguistics, pragmatics, speech technology, sociolinguistics and
 health communication, among others.

The term ``corpus'' has a slightly different usage in classical philology: they
designate a structured collection of texts, but that collection is not
primarily intended for the testing of linguistic hypotheses. Instead, we have,
for instance, the ancient corpus Tibullianum, or modern-day collection, for
  instance the Corpus Papyrorum Judaicarum, etc. We are primarily interested in
  the digital techniques used to create linguistic corpora; so let us first
  take a look at the progress of the digital classics.

The efforts began with the aforementioned Index Thomisticus, the first
computer-based corpus in a classical language; but the first true impetus was
the foundation of the Thesaurus Linguae Graecae project in 1972.
