%************************************************
\chapter{Adapting the DDBDP data}
\label{chp:adaptation}
\minitoc\mtcskip
%************************************************
\section{XSL transformations}
\label{sect:xslt}
The GitHub repository for the IDP does not only contain transcriptions
of papyri: it offers us the entire skeleton of the website, most
importantly for us its XSL stylesheets. These stylesheets allow us to
convert the EpiDoc XML of the raw files into a easily readable format of
our choosing. The set offered is exhaustive, well-designed and
customisable.

The files can be found in the directory papyri/navigator/pn-xslt. The
file global-parameters.xml should first be edited to accomodate one's
needs: it allows the user to choose between different styles for the
Leiden notation, metadata, line numbering, apparatus and edition. I have
included my own parameters file in the attached USB key.

I have chosen to change a few technical details to fit the necessary requirements
for an easily taggable corpus: I fixed the stylesheets for conversion to
plain text, as in their original state they caused errors upon
compilation; I have adapted certain aspects of the critical notation
in order to avoid any impediment for automated treatment; and lastly,
I have removed all line breaks, hyphens and space-filling elements.

\subsection{Technical issues}
\label{subsect:techissues}
The first was done by analysing the error message given by Saxon: the
error messages refer to missing variables and functions. I have applied
a quick fix using the following steps:

1. uncommenting lines 32-33 in papyri/navigator/pn-xslt/globals-varsandparams.xsl, i.e.:

\begin{lstlisting}
  <!--<xsl:param name="line-inc"
              select="document($param-file)//parameter[name = 'line-inc']/value"/>-->
\end{lstlisting}
becomes
\begin{lstlisting}
<xsl:param name="line-inc"
              select="document($param-file)//parameter[name = 'line-inc']/value"/>
\end{lstlisting}

2. removing any mention of the function EDF:f-wwrap in teispace.xsl and
teisupplied.xsl.

\subsection{Critical issues}
\label{subsect:criticalissues}

Perhaps flying in the face of usual practice, I have chosen to remove
most critical notation in order to facilitate the processing of the
text. Brackets, parentheses and the like have all flown out the window;
those acquainted with the difficulties involved in the critical edition
of papyri will very possibly be angered and, justly, maintain that
all conjectures found in editions of papyri are the
product of guesswork - very educated guesswork, artfully straddling the border
between intuition and exactitude, yet guesswork. The
reasoning, then, would be that one cannot base one's research on this;
shaky foundations, after all, entail a shaky building.

I will grant this objection, but a retort is certainly in order. We must
remind ourselves that the intent is not to deduce linguistic facts from
unique instances which also happen to be conjectures ; rather, the
point is to corroborate or disprove certain hypotheses using a relative
abundance of evidence. It is hard to imagine that a single conjecture
could convincingly contradict dozens of instances that paint a wholly
different picture; what's more, our interest lies mainly in language,
and it is hard to deny that the usual apple of discord when discussing
conjectures is not their grammaticality, but their semantic and
historical implications - that is, they are in most cases in agreement
with linguistic reality. The cases that are not, or are too much so,
will therefore inevitably be outnumbered by those that are. One might
add that the vast majority of conjectures usually involve lacunae of no
more than one or two characters (check?) and are thus sufficiently small
to be immune to overly zealous coniectores.

When considering this kind of purism, it is not hard to imagine some
saying that is it would be a good idea to use diplomatic editions in
order to convey an image of the text which is as authentic as possible.
The problems with such an approach are various, but the most important
would without a doubt be the word divisions. Parsing would be rendered
nigh-impossible without the development of new and complex algorithms;
combine this with the textual difficulties already present in modern edited
texts, and we are faced with an even greater problem. It seemed best to
me then, to make do with the easily available preprocessed material.

In tpl-reasonlost.xsl, I have replaced all instances of
\begin{lstlisting}
<xsl:text>[</xsl:text> 
\end{lstlisting}
and 
\begin{lstlisting}
<xsl:text>]</xsl:text> 
\end{lstlisting}
with
\begin{lstlisting}
<xsl:text></xsl:text>.
\end{lstlisting}

Doing this has the effect of removing all brackets from the equation. I have
also taken the decision of removing all expanded abbreviations from the texts,
for the simple reason that they serve a purely clarificatory purpose and were
intentionally written as abbreviations  To do this, one must edit the file
teiabbrandexpan.xsl and delete the contents of all templates - without deleting
the templates themselves, as this would cause an immediate error. The end
result may be found in the GitHub repository and should look like this:

\begin{lstlisting}
<?xml version="1.0" encoding="UTF-8"?>
<!-- $Id: teiabbrandexpan.xsl 1542 2011-08-22 18:09:22Z gabrielbodard $ -->
<xsl:stylesheet xmlns:xsl="http://www.w3.org/1999/XSL/Transform"
   xmlns:t="http://www.tei-c.org/ns/1.0" exclude-result-prefixes="t"  version="2.0">
   <!-- Contains templates for expan and abbr -->

   <xsl:template match="t:expan">
      <xsl:apply-templates/>
      <!-- Found in tpl-certlow.xsl -->
      <xsl:call-template name="cert-low"/>
   </xsl:template>

   <xsl:template match="t:abbr">
      <xsl:apply-templates/>
   </xsl:template>

   <xsl:template match="t:ex">
   </xsl:template>

   <xsl:template match="t:am">
   </xsl:template>
</xsl:stylesheet>
\end{lstlisting}


In teilb.xsl, I have deleted lines 26-31 and modified lines 50-51 to the
following:

\begin{lstlisting}
                 <xsl:choose>
                 <xsl:when
               test="(@break='no' or @type='inWord') and preceding::node()[1][not(local-name() = 'space' or local-name() = 'g'
               or @reason='lost')]
               and not(starts-with($leiden-style, 'edh'))
               and not($edition-type='diplomatic')">
               <xsl:text></xsl:text>
                 </xsl:when>
                 <xsl:otherwise>
                   <xsl:text> </xsl:text>
                 </xsl:otherwise>
               </xsl:choose>
\end{lstlisting}

\subsection{Metadata}
\label{subsect:metadata}

If we are to to be able focus our attention on a particular period, style or
local variation of Greek, we must use the metadata provided by the IDP files;
yet we cannot leave the metadata embedded in the files as we now have them,
since that would skew the results by the sheer amount of metadata - every text
generated by the XSLT stylesheets contains two lines of it!
