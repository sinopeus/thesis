
%************************************************
\chapter{Analysis}
\label{chp:analysis}
%\minitoc\mtcskip
%************************************************

\section{Scope of the problem}

This is the place to quantify the dimensions and complexity of the
problem at hand, as this clarifies choices we must make later on in
the design and implementation of our program. We summarise the major
component problems of our task facing us.

A first major obstacle is the size and diversity of the corpus. The
corpus of the papyri as presented on papyri.info contains more than
4.5 million words and spans nearly a millennium, by virtue of which it
inevitably contains tens of thousands of word forms which show the
evolution of the Greek language. This adds a great amount of
complexity: not only must our system recognise 'normal' word forms, it
must also be able to make inferences (albeit limited ones) about
unknown forms. 

The second major obstacle is the Greek language itself. Though it has
lost a great deal of morphological complexity in its evolution towards
its current state, the Greek of Hellenistic, Roman and Byzantine times
still possesses much more of it than does, say, English, which is the
target language for most research in natural language processing.

\section{Objectives}



\section{Methodology}