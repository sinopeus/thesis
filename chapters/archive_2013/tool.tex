%************************************************
\chapter{The tool}
\label{chp:design}
\minitoc\mtcskip
%************************************************
\section{Goals} % (fold)
\label{sec:tool-goals}

% section Goals (end)

\begin{enumerate} 
  \item The tool must be able to read EpiDoc XML and a subset of
      additional tags added for linguistic annotation. It should be,
      essentially, a text-based replica of the Papyrological Navigator as found
      at \texttt{papyri.info}. 
    \item search 
    \item statistics
    \item syntactic trees
\end{enumerate}

\section{Design} % (fold) \label{sec:tool-design} 

The tool works from the command line; while this type of interface has the
obvious disadvantage of being a type interface most people cannot work with
right off the bat, command line programs generally have an edge over programs
with a graphical user interface in some aspects. They are smaller, faster, more
efficient and flexible, and in Unix environments can easily be integrated in a
larger workflow; it is possible, for instance, to take the output of a program
and pipe it immediately into a text file or into another program for
processing, or to write a script which automates the use of the program.

% section Design (end)

\section{Technical} % (fold)
\label{sec:tool-technical}
The tool is written in Python 3. 

lxml for XML display?
Whoosh for indexing and search
numpy for statistics
CairoPlot for SVG syntactic trees
% section Technical (end)

\section{Interface} % (fold)
\label{sec:interface}

The basic command for launching the script is \texttt{python tjufy} from within
a terminal or simply \begin{center}\texttt{tjufy}\end{center} from within the
Python interpreter (from here on I will use this shorter notation). Running it
without options like this will display a help dialog; the same goes for running
\begin{center}\texttt{python tjufy -h}\end{center} or
\begin{center}\texttt{python tjufy -{}-help}\end{center}


% section User interface (end)

